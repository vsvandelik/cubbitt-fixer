\documentclass[12pt,a4paper]{article}
\usepackage[utf8]{inputenc}
\usepackage[T1]{fontenc}
\usepackage[czech]{babel}
\usepackage{graphicx}
\usepackage[left=1.50cm, right=1.50cm, top=1.50cm, bottom=1.50cm]{geometry}

\title{Specifikace ročníkového projektu}
\author{Vojtěch Švandelík}
\begin{document}
	\maketitle
	
	V následujícím dokumentu je uvedena specifikace ročníkového projektu na \textit{Matematicko-fyzikální fakultě Univerzity Karlovy}. Vedoucím ročníkového projektu je Mgr. Martin Popel, Ph.D. z Ústavu formální a~aplikované lingvistiky.
	
	\section{Cíl projektu}
	Obsahem ročníkového projektu je identifikovat chyby v automatickém překladu neuronového překladače \textit{CUBBITT} a~navrhnout způsoby jejich řešení.
	
	\section{Popis fází}
	\subsection{Identifikace chyb}
	Při hledání chyb se bude vycházet z korpusu \textit{CzEng 2.0}\footnote{http://ufal.mff.cuni.cz/czeng}. Tento obsahuje české i anglické texty přeložené prostřednictvím výše zmíněného překladače do opačného jazyka. V přeložených textech proběhne poloautomatické filtrování, které se pokusí potvrdit již dříve identifikované překladové problémy a rovněž se zaměří na hledání nových.
	
	Prozatím známé (a vcelku významné) překladové chyby jsou nevhodné překládání vlastních jmen osob a překlad jednotek u~čísel, kdy hodnota ztrácí faktický význam.
	
	\subsection{Hledání řešení}
	Prvotní fází napravování překladových chyb bude vkládání do textu speciální unicode znaky. Následně se bude zkoumat, zda-li překladač obalené části vět bude či nebude překládat. Dle získaných výsledků budou následovat další kroky. Jedním z nich by mohlo být přetrénování testovacích dat s ohledem na zamezení překládání speciálně označených tokenů.
	
	\subsection{Vyhodnocování výsledků}
	Úspěšnost vylepšení překladu bude vyhodnocována pomocí dostupných nástrojů -- např. metrika Bleu.
	
	\section{Výstup projektu}
	Výstupem ročníkového projektu by měla být rozvaha nad nalezenými překladovými chybami, společně s rešerší týkající se jejich oprav.
	
	V rámci implementační části by měl vzniknout základ evaluace výsledků a především by mělo dojít k využití alespoň některých návrhů řešení překladových chyb.
	
	Výstupy by měly posloužit jako podklady pro psaní bakalářské práce.
\end{document}